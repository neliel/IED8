\documentclass[a4paper, 11pt]{article}
%
\usepackage[utf8]{inputenc}
\usepackage[T1]{fontenc}
\usepackage{lmodern}
\usepackage[francais]{babel}
\usepackage[top=1cm, bottom=1cm, left=1cm, right=2cm]{geometry}
\usepackage{nopageno}
\usepackage{listings}
\usepackage{graphicx}
%
\newcommand{\env}[1]{\fbox{\begin{minipage}{\textwidth}#1\end{minipage}}}
%
\title{Exo Lsp 6 Prog}
\author{Sarfraz \bsc{kapasi}}
\date{04/11/2012}
%
\begin{document}
%
\maketitle
%
\section{Programme à faire}
\subsection{Question}
Partez de :
\begin{lstlisting}
(setq seau '(a b c d e f))
(setq bassine nil)
\end{lstlisting}
En vous inspirant de la partie sur un filtre, écrivez un programme qui verse du seau dans la bassine sans filtrer.
Qu'observez-vous dans la bassine ?
Que fait votre programme?
\subsection{Réponses}
\begin{lstlisting}
(loop
    (cond
        ((not seau) (return bassine)))
    (push (pop seau) bassine))
\end{lstlisting}
La bassine se rempli des éléments du seau à mesure que celui-ci se vide.
Le programme codé prends les éléments de la liste seau et en rends l'inverse dans la liste bassine.
%
\end{document}

