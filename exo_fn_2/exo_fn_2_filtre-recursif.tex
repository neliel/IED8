\documentclass[a4paper, 11pt]{article}
%
\usepackage[utf8]{inputenc}
\usepackage[T1]{fontenc}
\usepackage{lmodern}
\usepackage[francais]{babel}
\usepackage[top=1cm, bottom=1cm, left=1cm, right=2cm]{geometry}
\usepackage{nopageno}
\usepackage{listings}
%
\newcommand{\env}[1]{\fbox{\begin{minipage}{\textwidth}#1\end{minipage}}}
%
\title{Exo Fn 2 Filtre-récursif}
\author{Sarfraz \bsc{kapasi}}
\date{06/11/2012}
%
\begin{document}
%
\maketitle
%
\section{Une fonction qui filtre o dans une liste} % (fold)
\label{sec:Une fonction qui filtre o dans une liste}
\subsection{Question} % (fold)
\label{sub:Question}
Ecrire récursivement la fonction suivante :
\begin{lstlisting} % (fold)

    (filtre-o '(o o o ? o o o ! o o o))
    (o o o o o o o o o o)
\end{lstlisting}
% subsection Question (end)
\subsection{Réponse} % (fold)
\label{sub:Réponse}
\begin{lstlisting} % (fold)

    (defun filtre-o (liste)
        (cond
            ((not liste) nil)
            ((equal (car liste) 'o) (cons 'o (filtre-o (cdr liste))))
            (t (filtre-o (cdr liste)))))
\end{lstlisting}
% subsection Réponse (end)

% section Une fonction qui filtre o dans une liste (end)

\section{Une fonction qui filtre un élément donné dans une liste} % (fold)
\label{sec:Une fonction qui filtre un élément donné dans une liste}
\subsection{Question} % (fold)
\label{sub:Question}
Ecrire récursivement la fonction suivante :
\begin{lstlisting} % (fold)

    (filtre 'a '(a a a ? a ! a ? a a))
    (a a a a a a a)
\end{lstlisting}
% subsection Question (end)
\subsection{Réponse} % (fold)
\label{sub:Réponse}
\begin{lstlisting} % (fold)

    (defun filtre (liste elementAFilter)
        (cond
            ((not liste) nil)
            ((equal (car liste) elementAFiltrer) (cons elementAFiltrer (filtre (cdr liste))))
            (t (filte (cdr liste)))))
\end{lstlisting}
% subsection Réponse (end)

% section Une fonction qui filtre un élément donné dans une liste (end)

\section{Une fonction qui élimine tous les o d'une liste} % (fold)
\label{sec:Une fonction qui élimine tous les o d'une liste}
\subsection{Question} % (fold)
\label{sub:Question}
Ecrire récursivement la fonction suivante :
\begin{lstlisting} % (fold)

    (elimine-o '(x o o ! o d o ? o a b))
    (x ! d ? a b)
\end{lstlisting}
% subsection Question (end)
\subsection{Réponse} % (fold)
\label{sub:Réponse}
\begin{lstlisting} % (fold)

    (defun elimine-o (liste)
        (cond
            ((not liste) nil)
            ((equal (car liste) 'o) (elimine-o (cdr liste)))
            (t (cons (car liste) (elimine-o (cdr liste))))))
\end{lstlisting}
% subsection Réponse (end)

% section Une fonction qui élimine tous les o d'une liste (end)

\section{Une fonction qui élimine un élément donné d'une liste} % (fold)
\label{sec:Une fonction qui élimine un élément donné d'une liste}
\subsection{Question} % (fold)
\label{sub:Question}
Ecrire récursivement la fonction suivante :
\begin{lstlisting} % (fold)

    (elimine  'a '(x a a ! a d a ? a a b))
    (x ! d ? b)
\end{lstlisting}
% subsection Question (end)
\subsection{Réponse} % (fold)
\label{sub:Réponse}
\begin{lstlisting} % (fold)

    (defun elimine (liste elementAEliminer)
        (cond
            ((not liste) nil)
            ((equal (car liste) elementaEliminer) (elimine (cdr liste)))
            (t (cons elementAEliminer (elimine (cdr liste))))))
\end{lstlisting}
% subsection Réponse (end)

% section Une fonction qui élimine un élément donné d'une liste (end)
%
\end{document}

