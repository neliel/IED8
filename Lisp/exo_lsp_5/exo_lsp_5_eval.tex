\documentclass[a4paper, 11pt]{article}
%
\usepackage[utf8]{inputenc}
\usepackage[T1]{fontenc}
\usepackage{lmodern}
\usepackage[francais]{babel}
\usepackage[top=1cm, bottom=1cm, left=1cm, right=2cm]{geometry}
\usepackage{nopageno}
\usepackage{listings}
\usepackage{graphicx}
%
\newcommand{\env}[1]{\fbox{\begin{minipage}{\textwidth}#1\end{minipage}}}
%
\title{Exo Lsp 5 Eval }
\author{Sarfraz \bsc{kapasi}}
\date{\today}
%
\begin{document}
%
\maketitle
%
\section{Effets de set}
Voici une série d'évaluations de set, qui lient (ou tentent de lier) un nom à une valeur;
On supposera qu'elles ont toutes été faites dans le même contexte, avec les mêmes liaisons :
\begin{itemize}
    \item le nom foo lié à la valeur bar,
    \item le nom bar lié à la valeur zorglub,
    \item le nom zorglub lié à la valeur 5.
\end{itemize}
Dans chaque cas :
\begin{itemize}
    \item indiquez quel est le nom qu'on a tenté de lier ;
    \item indiquez comment se calcule la valeur qu'on a tenté de lier ;
    \item s'il y a une erreur (ou plusieurs) indiquez pourquoi (précisez le processus qui a provoqué l'erreur).
\end{itemize}
\begin{enumerate}
\item (set foo (cons (eval foo) bar))
  \begin{itemize}
  \item nom: bar
  \item valeur: (zorglub . zorglub)
  \end{itemize}
\item (set 'foo (cons 'foo (eval foo)))
  \begin{itemize}
  \item nom:
  \item valeur:
  \end{itemize}
\item (set bar (cons (eval bar) (eval foo)))
  \begin{itemize}
  \item nom: foo
  \item valeur: (foo . zorglub)
  \end{itemize}
\item (set foo (+ (eval foo) 2))
  \begin{itemize}
  \item nom: bar
  \item valeur: erreur, on essaie ici d'additionner zorglub et 2
  \end{itemize}
\item (set 'bar (+ (eval bar) 2))
  \begin{itemize}
  \item nom: bar
  \item valeur: 7
  \end{itemize}
\item (set (eval 'bar) (+ (eval (eval foo)) 10))
  \begin{itemize}
  \item nom: zorglub
  \item valeur: 15
  \end{itemize}
\end{enumerate}

\section{Construction et évaluation d'un programme}
\subsection{Question A}
Soit le bout de code suivant :
\begin{lstlisting}
(loop (cond ((not bassine) (return bassine))) (pop bassine))
\end{lstlisting}
Que fait ce bout de code?\\
Réponse: Ce bout de code vide et rends une bassine.
\subsection{Question B}
Ecrivez un programme qui écrive le programme donné en Question A, sous forme de liste, avec la fonction cons, et qui l'évalue.
\begin{lstlisting}
(eval (cons 'loop
            (cons (cons 'cond
                        (cons (cons (cons 'not
                                          (cons 'bassine nil) )
                                    (cons (cons 'return
                                                (cons 'bassine nil) )
                                          nil) )
                              nil) )
                  (cons (cons 'pop
                              (cons 'bassine nil)) nil)) ))
\end{lstlisting}
%
\end{document}
